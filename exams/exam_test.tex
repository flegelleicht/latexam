%!TEX TS-program = xelatex

%%%%%%%%%%%%%%%%%%%%%%%%%%%%%%%%
% DOCUMENT SETUP
%%%%%%%%%%%%%%%%%%%%%%%%%%%%%%%%
\documentclass[parskip=half, a4paper]{scrartcl}
\pagestyle{empty}

\usepackage[ngerman]{babel}
\usepackage[T1]{fontenc}
\usepackage{ae}
\usepackage{lmodern}

\setlength{\headheight}{2.5cm}

\usepackage{graphicx}
\usepackage{lastpage}
\usepackage{tabularx}

\usepackage[automark, footsepline, plainheadsepline]{scrpage2}
  \pagestyle{scrheadings}
  \clearscrheadings
  \clearscrplain
  \clearscrheadfoot
  \ihead[{\includegraphics[height=2cm]{images/logo.png}}]{{\includegraphics[height=2cm]{images/logo.png}}}
  \chead[~\~\ YOUR EXAM TITLE HERE]{~\\[1cm]\LARGE\bfseries\color{ci} YOUR EXAM TITLE HERE}
  \ofoot[Seite \thepage von \pageref{LastPage}]{Seite \thepage~von \pageref{LastPage}}

\usepackage{enumitem}
  \setlist{parsep=0pt,listparindent=\parindent}

\usepackage{xcolor}
  \definecolor{ci}{cmyk}{.29,0.85,0.56,0.47}

\usepackage{hyperref}
  \def\DefaultHeightofCheckBox{1.5ex}
  \def\DefaultWidthofCheckBox{\DefaultHeightofCheckBox}

\usepackage{array}
  \newcolumntype{L}[1]{>{\raggedright\let\newline\\\arraybackslash\hspace{0pt}}m{#1}}
  \newcolumntype{C}[1]{>{\centering\let\newline\\\arraybackslash\hspace{0pt}}m{#1}}

\usepackage{fontspec}
\setmainfont[Path = fonts/,
  UprightFont = *-Regular,
  BoldFont = *-Bold,
  ItalicFont = *-Italic]{Carlito}

\renewcommand*{\LayoutCheckField}[2]{% label, field
\leavevmode
#2 #1%
}


%%%%%%%%%%%%%%%%%%%%%%%%%%%%%%%%
% DOCUMENT START
%%%%%%%%%%%%%%%%%%%%%%%%%%%%%%%%

\begin{document}

  \renewcommand*{\DefaultOptionsofText}{print,bordercolor=black}
  \newcounter{FrageNummer}\setcounter{FrageNummer}{1}
  \newcommand{\DieFrageNummer}{\theFrageNummer.~\addtocounter{FrageNummer}{1}}

  \begin{Form}[action=mailto:your-email-here@example.com,encoding=html,method=post]

  \section{Allgemeines}
  \begin{center}
  Dies ist der Prüfungsbogen für \\
  {\bfseries YOUR EXAM TITLE HERE}
  \end{center}

  Bitte tragen Sie auf diesem ersten Blatt Ihren Namen und Ihre Anschrift ein.

  \begin{tabularx}{\textwidth}{lX}
      Name & \TextField[name=name, width=10cm,value={}]{}\\[1cm]
      Arbeitgeber & \TextField[name=arbeitgeber, width=10cm,value={}]{} \\[1cm]
  \end{tabularx}

  Auf den kommenden Seiten finden Sie Aufgaben, die Ihr Wissen über YOUR EXAM TITLE HERE überprüfen. {\bfseries Beachten Sie}: Alle Multiple-Choice-Fragen haben mindestens eine richtige Antwortmöglichkeit, es können zusätzlich aber weitere Antworten richtig sein. Eine Multiple-Choice-Frage gilt nur dann als richtig beantwortet, wenn alle richtigen Antwortmöglichkeiten ausgewählt wurden.

  Zum Bestehen der Prüfung sind mindestens 50\% der Fragen richtig zu beantworten.

  Nach Erhalt dieses Prüfungsbogens haben Sie 8 Stunden Zeit, um die Fragen zu beantworten. Schicken Sie uns innerhalb dieser 8 Stunden den ausgefüllten Fragebogen wieder per Mail an zurück. Zu spät eintreffende Antworten können wir nicht akzeptieren.

  Nach Bestehen der Prüfung erhalten Sie ...\\[2cm]

  Viel Erfolg bei der Prüfung!\\[0.5cm]
  \newpage\section{Thema Nummer 1}
    \begin{tabularx}{\textwidth}{lX}
      \hline\\[0.1cm] 
      {\large\DieFrageNummer} & {\large Welche der folgenden Antworten sind richtig?} \\[10pt]
        \CheckBox[name=t0q0a5]{} & Antwort 6 \\[6pt]
        \CheckBox[name=t0q0a1]{} & Antwort 2 \\[6pt]
        \CheckBox[name=t0q0a3]{} & Antwort 4 \\[6pt]
        \CheckBox[name=t0q0a2]{} & Antwort 3 \\[6pt]
        \CheckBox[name=t0q0a4]{} & Antwort 5 \\[6pt]
        & Was fällt Ihnen sonst noch ein?\\[10pt]
        & \TextField[multiline, name=q0_ff,width=0.9\textwidth, height=5cm,borderwidth=1,bordercolor=0 0 0,borderstyle=I,backgroundcolor=0.9 0.9 0.9,value={}]{}\\[10pt]
          \end{tabularx}

  \newpage\section{Thema Nummer 2}
    \begin{tabularx}{\textwidth}{lX}
      \hline\\[0.1cm] 
      {\large\DieFrageNummer} & {\large Welche der folgenden Antworten sind richtig?} \\[10pt]
    \multicolumn{2}{c}{\includegraphics[width=1.5cm]{images/example/image.png}} \\[10pt]
        \CheckBox[name=t1q1a9]{} & Antwort 3 \\[6pt]
        \CheckBox[name=t1q1a7]{} & Antwort 1 \\[6pt]
        \CheckBox[name=t1q1a10]{} & Antwort 4 \\[6pt]
        \CheckBox[name=t1q1a12]{} & Antwort 6 \\[6pt]
        \CheckBox[name=t1q1a8]{} & Antwort 2 \\[6pt]
    \end{tabularx}

%  EXAM KEY: tq0a5a1a3a2a4a6a0tq0a2a0a3a5a1a4
%  \Submit{Ergebnisse senden}
  \end{Form}
\end{document}
